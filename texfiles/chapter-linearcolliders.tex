\chapter{Future Linear Colliders}
[...]

\section{Introduction}
[...]

\section{The physics case for a lepton collider}
[...]

\section{The International Linear Collider}
[...]

As of writing, the initial centre-of-mass energy of the ILC is planned to be 380 GeV, with future upgrades increasing it to 500 GeV and finally 1 TeV.

The proposed site for the ILC is the Kitakami mountains in Iwate prefecture, Japan. [...]

\subsection{The ILD and SiD detectors}
One of the unique features of the ILC is the push-pull detector system. This is a moving platform in the chamber housing the interaction point, upon which two detectors can be mounted. The platform can be moved to change which detector is in the beamline, allowing a linear collider to function with multiple detectors. Switching detectors is expected to take [?] hours. This allows the two detectors to specialise for different physics studies [?], [...]

\subsubsection{The International Large Detector (ILD)}
[...]

\subsubsection{The Silicon Detector (SiD)}
[...]

\section{The Compact Linear Collider}
[...]