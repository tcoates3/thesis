\chapter{Future Colliders}
\label{chapter:colliders}

\epigraph{Progress is not a straight line.}{An Wang}

In the post-LHC era, particle physics is at somewhat of an impasse. The \acrfull{SM} has held up to most experiments and observation, and its predictive power is now exhausted. There are tantalising hints at a theory beyond the \acrlong{SM} -- in the form of CP-violation and dark matter -- but as of yet all new ways to probe the \acrshort{SM} for its weaknesses, to see the greater theory behind it, have yielded very little. 

There are many planned investigations to attempt to identify physics \acrlong{BSM} that use the \acrfull{LHC}, or plan to leverage the upgrades for the \acrfull{HL-LHC}. However, now that the Higgs boson has been identified successfully, one of the most fruitful avenues for further research is the construction and operation of a lepton collider at the energy frontier, with sufficient centre-of-mass energy to produce Higgs bosons in large numbers.

This is what motivates the several proposals for future lepton colliders around the world today, that would operate to complement and expand the reach of particle physicists beyond what the \acrshort{LHC} is currently capable of. These proposed lepton colliders are broadly split into two groups: linear colliders and circular colliders. 

Linear colliders use two accelerator arms pointed towards a single interaction point, and in general are capable of high centre-of-mass energies. The main candidates for this style of collider are the \acrfull{ILC} and the \acrfull{CLIC}. 

Circular colliders use a similar layout to the \acrshort{LHC}, with a circular accelerator capable of creating collisions at multiple different interaction points around the circumference of the ring, which permits multiple detectors. The downsides of a circular collider are that lighter particles like electrons are more prone to energy losses via synchrotron radiation, limiting the centre-of-mass energies that a circular lepton collider can operate at. However, in exchange they tend to have much higher luminosities, allowing greater numbers of collisions and higher yields of certain processes or channels. The main candidates for circular colliders are the \acrfull{FCC} and the \acrfull{CEPC}. 

These proposed colliders share many of the same motivations, design considerations, features, and challenges, as does the ongoing international effort in research and development to make these colliders a reality.

\section{The Standard Model}
The \acrlong{SM} of particle physics divides nature up into two domains: particles and forces. The four fundamental forces are electromagnetism, the strong interaction, the weak interaction, and gravitation\footnote{Gravitation is not contained within the \acrshort{SM}, although many \acrshort{BSM} models attempt to integrate it.}.

The particle domain is split into two -- the spin-half particles or \textit{fermions}, and the integer spin particles or \textit{bosons}. The fermions are further split into two -- the quarks and the leptons. These categories are defined by their relationship to the four funddamental forces: leptons can interact via the electroweak interaction only, while quarks can interact via both the electroweak and the strong interaction.

The bosons are a special case, as they do not simply interact via certain forces, but are the medium through which these forces propogate. For instance, when two particles undergo an electromagnetic interaction, they exchange photons ($\gamma$). Likewise, weak interactions involve the exchange of W or Z bosons, and strong interactions the exchange of gluons ($g$). For this reason they are also called the \textit{force carriers} or \textit{gauge bosons}.

A table summarising the particles of the \acrlong{SM} is shown in Fig. \ref{figure:colliders/SM}.

\begin{figure}[h]
	\centering
	\includegraphics[width=0.85\textwidth]{../Pictures/StandardModel.png}
	\caption{Table of particles in the Standard Model, each with their mass, charge and spin.}
	\label{figure:colliders/SM}
\end{figure}

\subsection{The \acrlong{SM} as a quantum field theory}
The \acrlong{SM} is conceived as a quantum field theory. This describes a number of quantum fields, whose values are defined over all space. In this conception of particle physics, all particles are in fact excitations of the underlying fields, called \textit{quanta}. So the \acrlong{SM} can be thought of as a system of several overlapping fields:

\begin{itemize}
	\item the \textit{fermion fields} ($\psi$), whose quanta are the fermions (e.g. the leptons and quarks)
	\item the \textit{electroweak fields}, whose quanta are the $W_1$, $W_2$, $W_3$, $B$ bosons 
	\item the \textit{gluon field} ($G_a$), whose quanta is the gluon
\end{itemize}

The interactions of these quantum fields can be described by the Lagrangian density $\mathcal{L}$, usually referred to simply as the Lagrangian.

We can mathematically formulate the \acrlong{SM} by writing out a Lagrangian expression for the \acrlong{SM} $\mathcal{L}_{SM}$. We can define this Lagrangian as a combination of individual Lagrangians for the fields described above:

\begin{equation}
	\mathcal{L}_{SM} = \mathcal{L}_{\psi} + \mathcal{L}_{EW} + \mathcal{L}_{QCD}
\label{eq:sm-lagrangian-1}
\end{equation}

where the $\mathcal{L}$ terms in order are the fermion or Dirac term, the electroweak term, and the quantum chromodynamic (or strong) term. 

However, a problem arises if we try to define the masses of the fermions in the fermion Lagrangian $\mathcal{L}_{\psi}$. This can be seen if we try to add a mass term and write it out fully:

\begin{equation}
	m \overline{\psi} \psi = m(\overline{\psi}_L + \overline{\psi}_R)(\psi_L + \psi_R) = m(\overline{\psi}_L \psi_R + \overline{\psi}_R \psi_L)
\label{eq:fermion-mass}
\end{equation}

This term then couples both to left-handed and right-handed fields. But these fields transform differently -- applying the same transformation to these fields has different results depending on the chirality of the field. The gauge symmetry is broken and the fermion mass term is \textit{not} invariant under electroweak symmetry -- the exact form depends on the choice of gauge. If we cannot write a mass term that is consistent with the theory and respects its symmetries, then it follows the fermions do not have mass. However this represents a clear conflict with empirical evidence. 

In the bosons we find a similar problem. We cannot write a mass term in the electroweak Lagrangian $\mathcal{L}_{EW}$, as that mass term will also break the gauge symmetry in a similar way. This presents no problem for the photon $\gamma$, which is known to be massless, but the other electroweak bosons -- the $W^+$, $W^-$ and $Z^0$ -- are known to be massive.

Therefore the \acrshort{SM} predicts that \textit{all} particles should be massless. Given that we know with certainty that all of the quarks, the charged leptons, and the W and Z bosons are massive, it follows that these particles do not have an intrinsic mass but instead acquire their mass from some interaction.

\subsection{The Higgs mechanism}
The problem of particle masses is solved by the introduction of the scalar Higgs field with four degrees of freedom:

\begin{equation}
	\phi = \frac{1}{\sqrt{2}} \begin{pmatrix}
	\phi^+ \\
	\phi^0
	\end{pmatrix}
\label{higgs-field}
\end{equation}

where the superscripts of $\phi$ denote electric charge. If we describe the Higgs field with a Lagrangian, we can separate it into a kinetic term and a potential term:

\begin{equation}
	\mathcal{L}_{H} = T_H - V(\phi)
\label{eq:lagrangian-higgs}
\end{equation}

The Higgs potential $V(\phi)$ is found to be of an unusual but characteristic shape -- often called the Mexican hat potential (see Fig. \ref{figure:colliders/mexican-hat}).

\begin{figure}[h]
	\centering
	\includegraphics[width=0.55\textwidth]{../Pictures/MexicanHatPotential.png}
	\caption{Diagram of the Higgs potential, or Mexican hat potential.}
	\label{figure:colliders/mexican-hat}
\end{figure}

If we think of the Higgs beginning at the origin -- at the apex of the sombrero -- then this potential is symmetrical. But the Higgs can `roll down' the slope into the trough to occupy a lower-energy state. Once this transition occurs, the potential is no longer symmetrical. This is the spontaneous symmetry breaking, and the result is that at low energies, the Higgs field has a nonzero vacuum expectation value $\braket{\phi^0} = v$.

It is notable that this is the \textit{only} parameter of the \acrlong{SM} that is not dimensionless -- the vacuum expectation value of the Higgs field is in units of mass, and thus sets the scales of all other masses in the \acrshort{SM}. 

Once the symmetry is broken, three of the four degrees of freedom of the Higgs field mix with the bosons of the electroweak fields, giving them mass. The fourth degree of freedom emerges as a scalar boson: the Higgs boson $H^0$.

Thus the W and Z bosons have masses that are defined in terms of the Higgs vacuum expectation value $v$ and the strength of their coupling to the Higgs field $g$:

\begin{equation}
	m_W = \frac{gv}{2}
\label{eq:w-mass}
\end{equation}
\begin{equation}
	m_Z = \frac{v}{2}\sqrt{g^2 + g'^2}
\label{eq:z-mass}
\end{equation}

This is the ``mixing'' of the degrees of freedom of the Higgs field -- the four degrees of freedom that would normally yield Goldstone bosons instead mix with the $W^+$, $W^-$, and $Z^0$ bosons, giving them mass. The fourth remaining degree of freedom emerges as the Higgs boson itself.

\subsection{The Yukawa coupling}
While the Higgs mechanism as described above explains the masses of the W and Z bosons, this does not explain the masses of the quarks and leptons. As we cannot add a consistent mass term to the fermion fields, and the degrees of freedom of the Higgs field do not mix with them, the fermions must acquire mass via some other mechanism.

This is solved by the Yukawa coupling. This is an interaction between the Higgs scalar field $\phi$ and the fermion field $\psi$ that takes the form

\begin{equation}
	V \approx g \overline{\psi} \phi \psi
\label{eq:yukawa-interaction}
\end{equation}

where $g$ is a coupling constant.

However, due to the spontaneous symmetry breaking of the Higgs mechanism, the Yukawa potential will have a minimum nonzero value $\phi_0$. The Higgs field can thus be thought of as a combination of the basic field term $\phi$ and a term representing the nonzero vacuum expectation value $\phi_0$:

\begin{equation}
	\tilde{\phi} = \phi - \phi_0
\end{equation}

If the Yukawa coupling (Eq. \ref{eq:yukawa-interaction}) is re-written using this formulation of the Higgs field, it becomes

\begin{equation}
	V \approx g \overline{\psi} \phi \psi - g \overline{\psi} \phi_0 \psi
\end{equation}

As $\phi_0$ is a constant, the second term instead becomes $g \phi_0 \overline{\psi} \psi$ which then simplifies to $g \phi_0$, resembling a mass term. Similarly to the W and Z bosons, fermion masses are then defined in terms of the Higgs vacuum expectation value $v$ and the strength of their coupling to the Higgs field $g$:

\begin{equation}
	m_i = -\frac{f_i v}{\sqrt{2}}
\end{equation}

where $i$ refers to the family of fermions -- either the charged leptons, up-type quarks, or down-type quarks.

Returning back to the Lagrangian in Eq. \ref{eq:sm-lagrangian-1}, instead of attempting to write a mass term, we instead add a Yukawa coupling Lagrangian $\mathcal{L}_{YU}$. This Lagrangian is gauge-invariant, avoiding the problem of a fermion mass term.

\subsection{The completed \acrlong{SM}}
In light of the necessity of the Higgs field, and the part it plays in generating the masses of the fermions and vector bosons via spontaneous symmetry breaking, we must then rewrite the \acrshort{SM} Lagrangian from Equation \ref{eq:sm-lagrangian-1} to include the effects of the Higgs field and the Yukawa coupling:

\begin{equation}
	\mathcal{L}_{SM} = \mathcal{L}_{\psi} + \mathcal{L}_{EW} + \mathcal{L}_{QCD} + \mathcal{L}_{H} + \mathcal{L}_{YU}
\label{eq:sm-lagrangian-2}
\end{equation}

This then constitutes a full Lagrangian density for all \acrlong{SM} fields.

%%		%%		%%		%%		%%
%Electroweak theory describes the unification of electromagnetism with the weak interaction at high energies, and was awarded the Nobel prize in 1979. One of the important features was electroweak symmetry, which describes the electroweak interaction as an $SU(2) \times U(1)$ gauge group. This then produces three bosons from the $SU(2)$ group ($W_1$, $W_2$, $W_3$), and one $B$ boson from the $SU(1)$ group. All of these are massless. The $W_{1,2,3}$ bosons correspond to the gauge bosons for the weak interaction ($W^+$, $W^-$, $Z^0$) and the $B$ boson to the photon, which mediates electromagnetism.

%However, experimental evidence shows that the W and Z bosons are \textit{not} massless. The $W^+$ and $W^-$ have the same mass of 80 GeV, while the $Z^0$ boson has a mass of 91 GeV.
%%		%%		%%		%%		%%
%\subsubsection{The Higgs mechanism and electroweak symmetry breaking}
%The solution to this problem comes in the form of the Higgs mechanism and electroweak symmetry breaking. This posits the existence of a scalar field permeating all of space -- the Higgs field. The potential of the Higgs over all space is found to be in the shape of the so-called `Mexican hat potential', with a high value at the origin, and a minimum value elsewhere. This means that the Higgs must `roll down' the slope, settling in the lower-energy regions. This in turn causes the Higgs to have a nonzero vacuum expectation value, and then spontaneous breaking of the electroweak symmetry.
%
%With the breaking of the electroweak symmetry, three of the Higgs' four degrees of freedom (which would normally yield Goldstone bosons) instead mix with the $W^+$, $W^-$, and $Z^0$ bosons, giving them mass. The remaining fourth degree of freedom emerges as \textit{the} Higgs boson. 
%
%The Higgs boson is an important probe of physics, as it couples to all massive particles in proportion to their mass. The top quark, being the most massive elementary particle, has the strongest coupling to the Higgs boson. This makes the top quark an extremely useful way to search the Higgs sector, as the strength of its coupling means that any divergences from the \acrshort{SM} in the Higgs sector will be most visible in the Higgs boson's interactions with the top quark. 

\section{The physics case for a lepton collider}
With the observation of a Higgs boson with a mass of 125 GeV, based on data from the \acrlong{LHC}, the Standard Model of particle physics is now functionally complete -- all of it's major predictions have been observed. This is a testament to it's quality as a theory, where it's predictive power and accuracy is one of the best in all of the sciences.

But despite this, a multitude of observations have shown that the Standard Model cannot be a complete theory of nature. Many phenomena have been observed that the Standard Model cannot predict, or that don't seem to interact with the Standard Model in any way.

Particle physicists are now forced to seek answers to three questions that the Standard Model cannot solve:

\begin{enumerate}
	\item What is dark matter? Astrophysics observations support the existence of a neutral, weakly-interacting substance that composes around 85\% of all mass in the universe. Yet this substance cannot be explained by any known form of matter, and is completely unprecedented by the Standard Model.
	\item Why is there so little antimatter? The symmetries inherent in the Standard Model predict that the Big Bang would have created an equal quantity of matter and antimatter. Yet the universe today is dominated by matter.
	\item Why does the Higgs field fill space and give mass to elementary particles? The existence of the Higgs field and the coupling of the Higgs boson to other particles can be understood from the Standard Model but their origin or cause is still unexplained.
\end{enumerate}

In order to answer these questions, new theories of physics Beyond the Standard Model (\acrshort{BSM}) have been made, and need to be experimentally tested. To do this, particle collider experiments at the energy frontier are needed. The \acrfull{LHC} has already been used extensively in searches for new physics, in the forms of new particles, rare and exotic decays, supersymmetry, and dark matter.

However, the running of a lepton collider at the energy frontier would be complementary to the \acrshort{LHC}'s continuing physics programme -- there are many events or channels that are inaccessible or difficult to examine in one environment that are much simpler or higher precision in the other. In this way, a lepton collider would help to improve and refine measurements already taken at the \acrshort{LHC}, while also allowing physicists to examine new channels and decays that were not accessible to it. A number of processes and their discovery potential can be seen in Table \ref{table:colliders/physics-goals}.

This follows a historical pattern in particle physics -- as the energy frontier advances, hadron colliders are used to discover new physics and new phenomena, followed by lepton colliders to examine these phenomena in higher precision.

\subsection{Higgs physics}
Of specific interest to searches at lepton colliders would be the Higgs boson itself. Many \acrshort{BSM} models predict differences from the Standard Model in the Higgs sector -- such as several Higgs bosons with different masses, composite Higgs, charged Higgs etc. The comparatively `quiet' environment of a lepton collider allows higher precision measurements of the properties of the Higgs boson, placing better constraints on the presence of new physics. In addition, lepton colliders can easily operate at specific thresholds and ``hot spots'' for Higgs production, permitting a much greater number of events yielding Higgs bosons, and thus a greater sample to examine.

% First confirmation of h->bb decays paper: https://atlas.web.cern.ch/Atlas/GROUPS/PHYSICS/CONFNOTES/ATLAS-CONF-2018-036/
Additionally, the most common decay of the Higgs boson, at a branching ratio of 57.7\%, is the $H \rightarrow b \overline{b}$ process. Despite the high branching ratio, the huge \acrshort{QCD} backgrounds in a hadron collider have made this decay incredibly difficult to observe at the LHC -- in fact, the $H \rightarrow b \overline{b}$ decay has only fairly recently been experimentally confirmed by the \acrshort{ATLAS} experiment. However, in a lepton collider these \acrshort{QCD} backgrounds are significantly smaller, meaning this decay channel is much easier to analyse, opening up a huge number of Higgs events for analysis and examination.

Another highly interesting process uniquely accessible to lepton colliders is the higgstrahlung reaction: $e^+ e^- \rightarrow Zh$. The decay of the Z boson into lepton pairs $e^+ e^-$ or $\mu^+ \mu^-$ allows high precision kinematic measurements of the process without directly measuring the Higgs boson itself. This allows unprecedented measurement of the Higgs mass, while also making measurements of missing energy possible. If the Higgs boson has invisible decays -- such as dark matter particles or other undiscovered particles that don't couple to the \acrshort{SM} -- the higgstrahlung process allows their existence to be identified.

An additional benefit of the higgstrahlung process is that the recoiling Z boson can be used to identify all the Higgs decay modes, and thus direct measurement of Higgs couplings. With this understanding of the Higgs couplings and their rates and branching ratios, it is possible to perform a model-independent determination of the total rate of Higgs decay $\Gamma\textsubscript{h}$. This result then allows calculation of the absolute size of all other Higgs couplings.

In addition, the production of top quark pairs in combination with a Higgs boson will allow direct measurement of the top-Higgs Yukawa coupling. This is the strongest Higgs coupling in the \acrshort{SM}, and as such is most sensitive to new physics. A study examining the top-Higgs Yukawa coupling in detail can be found in Chapter \ref{chapter:analysis} of this thesis.

\subsection{Top physics}
In addition to the Higgs, the top quark is of high interest to possible physics programmes at a future lepton collider. Since it's discovery at Fermilab in 1995, there have been no lepton colliders operating with sufficient energy to produce top quarks -- unlike the charm and bottom quarks, which were studied in further detail in lepton colliders after their discovery in hadron machines. The production threshold for top quarks is 350 GeV, and no lepton colliders capable of this energy have been constructed, meaning that the advantages of a lepton collider have never been utilised to understand the top quark in greater detail. 

The detailed study of top quarks and their properties that would be made possible by a lepton collider would have many benefits, primarily in the precision of the measurements. Similar to the Higgs, low \acrshort{QCD} backgrounds make top quark events easier to reconstruct. The fact that the collision is between elementary particles, rather than a parton-parton collision as in a hadron collider, also permits are more well-defined centre of mass energy, which is not possible in hadron colliders.

Precision measurement of the top quark is an important test of the \acrshort{SM} -- many \acrshort{BSM} models envision their additional particles as partners of the top quark. The upshot of this is that because of this, these theories propose properties of the top quark that diverge from the SM. In-depth and precise analysis of the top quark's properties would place limits on these theories.

\subsection{Supersymmetry}
\acrfull{SUSY} is one of several candidate models for \acrshort{BSM} physics, and currently one of the most widely-researched and well-motivated. A large programme of searches for supersymmetric particles or signals at the \acrshort{LHC} is ongoing, though as of writing no new particles or signals of supersymmetry have been confirmed.

For the same reasons that lepton colliders provide ideal environments for doing precision Higgs and top physics, they also provide ideal conditions for supersymmetry searches. One example is searches for the first- and second-generation sleptons $\tilde{e}$ and $\tilde{\mu}$ that decay by the process
$$
		e^+ e^- \rightarrow \tilde{e_R^+} \tilde{e_R^-} \rightarrow e^+ e^- \tilde{\chi_0^1} \tilde{\chi_0^1} \\
$$
$$
		e^+ e^- \rightarrow \tilde{\mu_R^+} \tilde{\mu_R^-} \rightarrow \mu^+ \mu^- \tilde{\chi_0^1} \tilde{\chi_0^1}
$$

This requires high-efficiency reconstruction of leptons, as well as the usage of \acrfull{BDT} to distinguish the process from \acrshort{SM} processes and \acrshort{SUSY} backgrounds. 

A variety of similar processes involving supersymmetric particles would be available to study at a high-energy lepton collider, using the higher-precision detectors and lower backgrounds to improve current limits. Precise measurement of these processes and their cross-sections, as well as the masses of any resulting particles, will help to provide evidence for or put limits on supersymmetric theories.

\begin{table}[b]
\centering
	\begin{tabular}{ r  c  l }
	\hline \hline
	\textbf{Energy (GeV)} & \textbf{Reaction} & \textbf{Physics goal} \\ \hline
	 91 & $e^+ e^- \rightarrow Z$ & ultra-precision electroweak \\ \hline
	 160 & $e^+ e^- \rightarrow WW$ & ultra-precision W mass \\ \hline
	 250 & $e^+ e^- \rightarrow Zh$ & precision Higgs couplings \\ \hline
	 350-500 & $e^+ e^- \rightarrow t\overline{t}$ & top quark mass and coupling \\
	   & $e^+ e^- \rightarrow WW$ & precision W coupling \\
	   & $e^+ e^- \rightarrow \nu \overline{\nu} h$ & precision Higgs couplings \\ \hline
	 500 & $e^+ e^- \rightarrow f \overline{f}$ & precision search for $Z^\prime$ \\
	   & $e^+ e^- \rightarrow t \overline{t}h$ & top-Higgs Yukawa coupling \\
	   & $e^+ e^- \rightarrow Zhh$ & Higgs self-coupling \\
	   & $e^+ e^- \rightarrow \widetilde{\chi} \widetilde{\chi}$ & supersymmetry searches \\
	   & $e^+ e^- \rightarrow AH, H^+, H^-$ & extended Higgs states \\ \hline
	 700-1000 & $e^+ e^- \rightarrow \nu \overline{\nu} hh$ & Higgs self-coupling \\
	   & $e^+ e^- \rightarrow \nu \overline{\nu} VV$ & composite Higgs \\
	   & $e^+ e^- \rightarrow  \nu \overline{\nu} t \overline{t}$ & composite Higgs and top \\
	   & $e^+ e^- \rightarrow \tilde{t} \tilde{t}^*$ & supersymmetry searches \\ \hline \hline
	\end{tabular}
	\caption{Physics processes of interest at lepton colliders up to 1 TeV.}
	\label{table:colliders/physics-goals}
\end{table}

\begin{table}[b]
\centering
%	\begin{tabular}{l l l}
	\begin{tabular}{p{0.35\linewidth}  p{0.25\linewidth} p{0.35\linewidth}}
	\hline \hline
	\textbf{Process} & \textbf{\acrshort{HL-LHC}} & \textbf{\acrshort{CLIC}} \\ \hline

	Heavy Higgs scalar mixing angle sin\textsuperscript{2}$\gamma$ & $<$ 4\% & $<$ 0.24\% \\
	Higgs self-coupling $\Delta \lambda$& ~50\% at 60\% CL & [-7\%, +11\%] at 68\% CL\\
	BR(H $\rightarrow$ invisible) &  & $<$ 0.69\% at 90\% CL\\ \hline

	Higgs compositeness scale $m_*$ & $m_* >$ 3 TeV & Discovery up to $m_*$ = 10 TeV\\
	 & ($>$ 7 TeV for $g_* \simeq 8$) & (40 TeV for $g_* \simeq 8$) \\ \hline

	Top compositeness scale $m_*$ &  & Discovery up to $m_*$ = 8 TeV \\
	 &  & (20 TeV for small coupling $g_*$) \\ \hline

	Higgsino mass & $>$ 250 GeV & $>$ 1.2 TeV\\
	Slepton mass &  & Discovery up to ~1.5 TeV\\
	RPV wino mass &  & $>$ 1.5 TeV (0.03 m $< c\tau <$ 30 m)\\ \hline

	$Z^{'}$ (\acrshort{SM} couplings) mass & Discovery up to 7 TeV & Discovery up to 20 TeV \\ \hline

	NMSSM scalar singlet mass & $>$ 650 GeV (tan$\beta$ = 4) &  $>$ 1.5 TeV (tan$\beta$ = 4)  \\
	Twin Higgs scalar singlet mass & $m_\sigma = f >$ 1 TeV & $m_\sigma = f >$ 4.5 TeV \\ \hline

	Relaxion mass & $<$ 24 GeV & $<$ 12 GeV (all for vanishing sin$\theta$)\\
	Relaxion mixing angle sin\textsuperscript{2}$\theta$&  & $\leq$ 2.3\% \\ \hline

	Neutrino Type-2 see-saw triplet &  & $>$ 1.5 TeV (for any triplet vev)) \\
	 & & $>$ 10 TeV (for any triplet Yukawa coupling $\simeq$ 0.1) \\ \hline

	Inverse see-saw RH neutrino &  & $>$ 10 TeV (for Yukawa coupling $\simeq$ 1) \\ \hline

	Scale $V^{-1/2}_{LL}$ for LFV ($\overline{e}e$)($\overline{e}\tau$) & & $>$ 42 TeV \\ \hline \hline
	\end{tabular}
	\caption{Limits on new physics for the \acrlong{CLIC}, as compared with the \acrfull{HL-LHC}. The given sensitivities assume the full \acrshort{CLIC} physic programme covering the three centre of mass energies 380 GeV, 1.4 TeV and 3 TeV, with integrated luminosities of 1 \textsuperscript{-1}, 2.5 \textsuperscript{-1} and 5 ab\textsuperscript{-1} respectively. All limits are at a 95\% confidence level (CL) unless stated otherwise. Values taken from \cite{clic-yellow}.}
	\label{table:colliders/precisions}
\end{table}

%\begin{figure}[p]
%	\centering
%	\includegraphics[width=0.75\textwidth]{../Pictures/SimulatedEvent1.png}
%	\caption{}
%	\label{figure:colliders/ILC/blah}
%\end{figure}

\section{The International Linear Collider}
The \acrfull{ILC} is a proposed high-luminosity linear electron-positron collider based upon 1.3 GHz superconducting radio frequency (\acrshort{SCRF}) accelerating technology. The ILC would have a centre of mass energy of 250 GeV in the initial stage, upgradable to 500 GeV and then to 1 TeV at a later date, with a luminosity of $3.6\times 10^{34}$ cm\textsuperscript{-2} s\textsuperscript{-1} and using magnets with an accelerating gradient of 31.5 MVm\textsuperscript{-1} in metre-long superconducting nine-cell niobium cavities operating at 2K. The total footprint of the complex, including both accelerator arms, the storage rings, and the interaction point, would be 31 km in length. Other parameters of the \acrshort{ILC} can be seen in Table \ref{table:colliders/parameters}.

\begin{figure}[h]
	\centering
	\includegraphics[width=0.75\textwidth]{../Pictures/ILC-Schematic.jpg}
	\caption{View of the accelerator complex for the \acrlong{ILC}, showing the two linacs and storage and damping rings, with football field for scale.}
	\label{figure:colliders/ILC/main}
\end{figure}

One of the unique features of the \acrshort{ILC} is the ``push-pull'' detector system. This is a moving platform in the chamber housing the interaction point (\acrshort{IP}), upon which two detectors can be mounted. The platform can be moved to exchange which detector is in the interaction point, allowing a linear collider to function with multiple detectors. This allows the two detectors to specialise for different physics studies and goals, much like the various experiments at the \acrshort{LHC} at \acrshort{CERN}, which would normally not be possible with a linear collider.

There were a number of proposed sites for the \acrshort{ILC}, including Fermilab in the United States, \acrshort{CERN} in Geneva, \acrshort{DESY} in Hamburg, and \acrshort{JINR} near Moscow. The most recent possible site, which has had significant attention and planning devoted to it, is the Kitakami Highlands region of Iwate prefecture in Japan. This would be a greenfield site, located on the side of a mountain range and requiring that a significant amount of the facility's infrastructure be located underground, in tunnels dug within the granite rock of the region. 

The \acrshort{ILC} project published a \acrfull{TDR} in 2013 with extensive details of the technology and ongoing research and development into making the collider a reality. As the location of the collider has yet to be determined, there is currently no estimated dates for the construction of the \acrshort{ILC}. However according to a recent report\cite{ilc-timeline-2019}, if the Japanese government were to approve hosting the \acrshort{ILC} and the project was listed as high priority in the report from the European Strategy Update in May 2020, then a period of at least four years would be needed for detailed planning and international negotiations. Following this, there would be a ten-year period for construction and commissioning of the collider, which would result in first physics in the early- to mid-2030s, around the same time as the \acrshort{HL-LHC} will be concluding it's own physics programme.

However, a report from the Science Council of Japan (a representative organisation of the Japanese science community) released in early 2019 expressed that they had not reached a consensus as to whether to support hosting the \acrshort{ILC} in Japan. Some  of the reasons cited were concerns over international cost-sharing in the long-term, as well as whether the expected scientific outcomes would justify the unprecedented human resource requirements and infrastructure necessary to make the \acrshort{ILC} a reality\cite{linearcolliders-scj-report}.

%Reference (24/04/2019): http://www.linearcollider.org/content/decision-international-linear-collider-“not-what-we-had-hoped-progress-nevertheless” 
On 7th March 2019, the Japanese government released a statement to say that they would not be making a proposal to host the collider\cite{japan-ilc-decision}. They however expressed a keen interest in the future of the project, and that they would be continuing to contribute towards the research and development of the \acrshort{ILC}.

\subsection{ILC detectors}
\label{section:ILC-detectors}
Detector design for the \acrshort{ILC} is driven by the requirements of the physics programme -- many of the physics goals and targeted processes are highly dependent upon hadronic states, so precise jet reconstruction and high jet energy resolution is critical to meeting the expectations placed upon the \acrshort{ILC}. 

The only technique thought to be capable of delivering the necessary level of accuracy is \textit{particle flow}. Particle flow is an integrated approach which uses algorithms to examine the flow of energy through all parts of the detector to correlate different signals together to generate particle flow objects (\acrshort{PFO}s). The requirements for the use of particle flow is a very good separation of charged and neutral particles, translating into a need for high-efficiency trackers, and calorimeters capable of high-efficiency reconstruction of neutral particles. The design of the detectors for the \acrshort{ILC} has proceeded from these requirements.

\begin{figure}[!b]%
	\centering
    \subfloat{{\includegraphics[width=0.45\textwidth]{../Pictures/Colliders/CLIC-WZ-separation-NoBg.png} }}%
    \qquad
	\subfloat{{\includegraphics[width=0.45\textwidth]{../Pictures/Colliders/CLIC-WZ-separation-Bg.png} }}%
    \caption{Distributions of invariant mass of the W and Z boson from a studying into the effects of particle flow algorithms at the \acrlong{CLIC}, showing a clear separation of the W and Z boson masses \cite{particle-flow-clic}.}%
    \label{figure:colliders/particle-flow}%
\end{figure}

The need for large separations between charged and neutral particles requires that the vertex detector, tracker, and calorimeter systems are contained within a magnetic field, so that charged particles' path is curved by the field. In general, this separation depends on the physical size of the detector and the strength of the magnetic field. Thus there are essentially two basic approaches to detector design: a large detector with a lower magnetic field, using the size of the detector to separate charged particles from neutral particles; or a more compact detector utilising a much stronger magnetic field to create the same separation.

These two approaches are shown in the two detector concepts for the \acrshort{ILC} -- the \acrlong{ILD} using a 3.5 T magnetic field, and the more compact \acrlong{SiD} using a much stronger 5 T magnetic field. These two detectors will be discussed in detail below.

\subsubsection{The International Large Detector (ILD)}
% https://indico.cern.ch/event/765096/contributions/3295752/attachments/1785276/2906298/ILD_European_Strategy_Document.pdf
% Paper on Particle Flow: https://arxiv.org/pdf/0907.3577.pdf

The \acrfull{ILD} is a detector concept for the \acrshort{ILC} intended as a multi-purpose detector, with a strong focus on optimising the performance of particle flow algorithms as much as possible. To attain this, it uses several technologies for very high-resolution and high-efficiency tracking, as well as highly-granular calorimeters.

For vertex tracking, the \acrshort{ILD} uses three double-layers of pixel detectors using monolithic active pixel sensor (\acrshort{MAPS}) technology, with a spatial resolution of 4 \textmu m and a timing resolution of 2-4 \textmu s. For tracking, the \acrshort{ILD} uses a hybrid system, combining a gaseous time projection chamber (\acrshort{TPC}) with silicon detector layers placed both inside and outside the \acrshort{TPC} volume. This combination allows a high tracking efficiency with a low material usage.

The calorimeter system must be highly granular in order to best utilise particle flow. The calorimeters are split into the electromagnetic calorimeter (\acrshort{ECAL}) and hadronic calorimeter (\acrshort{HCAL}). The \acrshort{ECAL} utilises a silicon diode sampling calorimeter with diode pads of 5 $\times$ 5 mm\textsuperscript{2}. An option for an ECAL using thin scintillator strips is also being investigated. The \acrshort{HCAL} has two possible options available. The \acrfull{AHCAL} uses silicon photomultipliers (\acrshort{SiPM}) on tiles of plastic scintillator with a resolution of 3 $\times$ 3 cm\textsuperscript{2} using a fully analogue readout. The \acrfull{SDHCAL} uses \acrfull{RPC} with a higher granularity of 1 $\times$ 1 cm\textsuperscript{2}, but does not use a fully analogue output, meaning that amplitude information is more limited. 

These detectors are placed within a solenoid capable of generating a 3.5 T magnetic field, and then within an iron flux return yoke which is instrumented for muon identification and tail catching, as well as providing structural support for the detector. The finished ILD is expected to weigh 14,000 metric tonnes.

%% Image found here: http://cds.cern.ch/record/2637370/plots
%\begin{figure}[h]
%	\centering
%	\includegraphics[width=0.75\textwidth]{../Pictures/ILDQuadrantView.png}
%	\caption{Quadrant view of the \acrshort{ILD} components.}
%	\label{figure:colliders/ILD/ILD-quadrant}
%\end{figure}
%
%% Image found here: https://flc.desy.de
%\begin{figure}[h]
%	\centering
%	\includegraphics[width=0.75\textwidth]{../Pictures/ILDCutaway.jpg}
%	\caption{Rendering of the finished \acrshort{ILD}, cutaway to show the interal features, with human for scale.}
%	\label{figure:colliders/ILD/ILD-cutaway}
%\end{figure}

\begin{figure}[h]
	\centering
	\includegraphics[width=0.75\textwidth]{../Pictures/SimulatedEvent1.png}
	\caption{Visualisation of a simulated $e^+ e^- \rightarrow t \overline{t} h$ event in the \acrshort{ILD}. Charged particles can be easily identified by the curved or spiral paths they take within the magnetic field, and the jets are visible as the light pink and purple areas near the beampipes on either side.}
	\label{figure:colliders/ILD/tth-simulation}
\end{figure}

\subsubsection{The Silicon Detector (SiD)}
The \acrfull{SiD} is a detector concept for the \acrshort{ILC} that uses primarily silicon-based technology, with the aim to reduce cost while still maintaining high performance and attaining the \acrshort{ILC}'s physics goals. The \acrshort{SiD} is also more compact than the \acrshort{ILD}, utilising a stronger magnetic field to compensate.

The vertex and tracker systems are both composed of silicon sensors, using a cylindrical configuration. The vertex uses silicon pixel sensors while the tracker uses silicon strip sensors, both designed to be used with power pulsing -- the electronics are only powered and active when it is known that bunches will be colliding. This reduces power and cooling requirements. 

The high-granularity calorimeters are both nested within the barrel, inside the magnetic field. The \acrshort{ECAL} uses thirty alternating layers of tungsten absorber and silicon active layers, in 3.5 $\times$ 3.5 mm\textsuperscript{2} hexagonal pixels. The \acrshort{HCAL} uses alternating layers of steel absorber and a glass resistive plate chamber, with cells of 10 $\times$ 10 mm\textsuperscript{2}

Outside of the calorimeter, the superconducting solenoid generates a 5 T magnetic field, which enables the more compact detector design -- a higher magnetic field increases the spatial separation between charged and neutral particles, which is necessary for usage of particle flow algorithms.

Outside of the magnetic field is an iron flux return yoke, which similarly to the \acrshort{ILD} concept also acts as a structural support and is instrumented for muon identification and tail catching. 

\begin{figure}[p]%
	\centering
    \subfloat{{\includegraphics[width=0.45\textwidth]{../Pictures/ILDCutaway.jpg} }}%
    \qquad
	\subfloat{{\includegraphics[width=0.45\textwidth]{../Pictures/ILDQuadrantView.png} }}%
    \caption{Rendering of the finished \acrshort{ILD} cutaway to show the internal features (left); and a quadrant view of the \acrshort{ILD} components (right).}%
    \label{figure:colliders/ILD/double}%
\end{figure}

\begin{figure}[p]%
	\centering
    \subfloat{{\includegraphics[width=0.5\textwidth]{../Pictures/SiDCutaway.jpg} }}%
    \qquad
	\subfloat{{\includegraphics[width=0.36\textwidth]{../Pictures/SiDQuadrant.png} }}%
    \caption{Isometric view of the finished \acrshort{SiD} cutaway to show the internal features (left); and a quadrant view of the \acrshort{SiD} components (right).}%
    \label{figure:colliders/ILD/double}%
\end{figure}

\section{The Compact Linear Collider}
The \acrfull{CLIC} is a proposed linear electron-positron collider that would be located at \acrshort{CERN} in Geneva, Switzerland. The accelerator is a staged design, with the initial stage having a centre of mass energy of 380 GeV, focusing on precision measurements of top quark and Higgs physics. The further stages would increase the centre-of-mass energy to 1.5 TeV, then finally 3 GeV. Physics goals in these later stages would involve searches for new physics processes, as well as precision measurements of rare Higgs processes, and of new states discovered at the LHC or earlier stages of \acrshort{CLIC}. 

\begin{figure}[h]
	\centering
	\includegraphics[width=0.8\textwidth]{../Pictures/CLIC-Render.jpg}
	\caption{View of the accelerator complex for the \acrlong{CLIC}, showing the main beams and drive beams, and the interaction region.}
	\label{figure:colliders/CLIC/main}
\end{figure}

To attain these extremely high energies, the \acrshort{CLIC} accelerators will be able to produce accelerating gradients as high as 100 MVm\textsuperscript{-1}. These cannot be achieved with traditional klystrons due to the extremely high peak \acrshort{RF} power needed. Klystrons capable of providing this would be inefficient and prohibitively expensive, so the \acrshort{CLIC} accelerators plan to use a two-beam acceleration scheme. A drive beam is used, with low intensity but high power. \acrshort{RF} power is then extracted from the drive beam to power the main accelerator. In this way, the high power, shorter duration pulses can be achieved more economically, allowing the high acceleration gradients that will allow each arm of \acrshort{CLIC} to accelerate electrons up to 1.5 TeV in only 21 km. Other parameters of the \acrshort{CLIC} can be seen in Table \ref{table:colliders/parameters}.

The proposed site for the \acrshort{CLIC} experiment would be at \acrshort{CERN}, built beneath the existing \acrshort{LHC} ring and stretching across the French-Swiss border, running parallel to the feet of the Jura mountain range. This placement is determined by the geological features of the region around Geneva and the feet of the Juras, where the tunnels would be dug into the sedimentary `molasse' rock in the area. 

As of writing, the \acrshort{CLIC} project released a \acrfull{CDR} in 2010 detailing the current state of planning for the experiment. It has also submitted input to the European Particle Physics Strategy Update, which will decide which projects the \acrshort{CERN} collaboration chooses to pursue from 2020 onwards.

If given the go-ahead at the European Strategy Meeting, there would be a five-year preparation phase from 2020-2025 for finalising development of hardware, technical proposals, industrial procurement, site authorisation, etc. Construction would then take place from 2025-2034, with first beams in 2035.

The detectors envisioned for \acrshort{CLIC} are similar in design to those for the \acrshort{ILC}, and are usually referred to as CLIC\textunderscore ILD and CLIC\textunderscore SiD. See \ref{section:ILC-detectors} for a detailed description of these detector concepts.

\section{The Future Circular Collider}
The \acrfull{FCC} is a series of concepts for a future collider that would be located in the Geneva area near the existing LHC ring. The \acrshort{FCC} project as a whole has three different accelerator concepts -- the FCC-hh for proton-proton and ion-ion collisions, the FCC-ee for electron-positron collisions, and the FCC-he for electron-proton collisions.

The initial proposal is to construct a circular electron-positron collider -- the FCC-ee -- with a circumference of 100 km and delivering a maximum centre-of-mass energy of 400 GeV. The motivation for this is that at this energy range -- the electroweak scale -- the \acrshort{FCC} would be able to access the Z pole, the W- and top-pair production thresholds, as well as producing a large number of Higgs bosons. Other parameters of the \acrshort{FCC} can be seen in Table \ref{table:colliders/parameters}.

Unlike linear colliders, staged energy increases are not part of the FCC-ee plan. The usage of low-mass particles like electrons in a circular collider results in a high energy loss due to synchrotron radiation, which must be mitigated by the constant addition of energy. Significantly higher energies than those planned by the FCC-ee are not currently practical due to these losses. Instead, the FCC-ee would concentrate on a physics programme in the 90-365 GeV energy range, leveraging the much higher luminosities available to a circular collider compared to linear colliders in the same energy range.

A further part of the proposal for the \acrshort{FCC} is that following the conclusion of the physics programme of the FCC-ee, the tunnels and infrastructure would be re-used for the FCC-hh, a hadron collider. This follows in the footsteps of the \acrshort{LHC}, which was constructed in tunnels originally built to house the \acrfull{LEP}. It is claimed that the FCC-hh built in these tunnels would be able to reach centre-of-mass energies of at least 100 TeV.

According to the given timeline, the FCC-ee would begin construction in 2028, and first physics would take place in 2039. 

\begin{figure}[t]
	\centering
	\includegraphics[width=0.95\textwidth]{../Pictures/FCC-Scale.jpg}
	\caption{The ring of the proposed \acrlong{FCC} laid over satellite imagery of the region around the French-Swiss border at Geneva. The current accelerators are shown in grey for scale.}
	\label{figure:colliders/FCC/scale}
\end{figure}

%In general, the centre of mass energy of circular colliders is limited by losses to due synchrotron radiation. The energy loss per turn scales with $\frac{E^4}{m^4}$, meaning that lighter particles like electrons suffer more losses due to synchrotron radiation than heavier particles like hadrons. 

\section{The Circular Electron Positron Collider}
% http://cepc.ihep.ac.cn/CEPC_CDR_Vol2_Physics-Detector.pdf
The \acrfull{CEPC} is a proposal for a circular electron-positron collider with a circumference of 100 km that would be hosted in China, with a centre of mass energy of 240 GeV to operate as a Higgs factory. It is also intended to operate at 91 GeV and 160 GeV to produce large numbers of Z and W bosons. This lower centre of mass energy is due to considerations of energy loss via synchrotron radiation. However, the \acrshort{CEPC} collaboration intends to utilise these to produce several synchrotron radiation light sources, including two gamma-ray beamlines. Other parameters of the \acrshort{CEPC} can be seen in Table \ref{table:colliders/parameters}.

The \acrshort{CEPC} project released a \acrfull{CDR} in 2018 detailing the current state of planning for the experiment. It has also been submitted as input to the European Particle Physics Strategy Update. The current timeline envisions a five-year research and development period from 2018-2022, and construction to start in 2022, completing in 2030. The physics programme is expected to last for ten years, concluding in 2040. 

Similarly to the \acrshort{FCC}, it is expected that technology for superconducting magnets at high field strengths will have developed sufficiently to allow the tunnels to be used to house a high-energy hadron collider, called the \acrfull{SPPC}. 

The detectors of the \acrshort{CEPC} as outlined in the \acrshort{CDR} are very similar in design to the \acrshort{ILD}, in the design of using a silicon vertex tracker, a \acrshort{TPC} tracker, and similar options for the calorimeters. An additional difference is the optioning of a dual-readout calorimeter to replace the electromagnetic and hadronic calorimeters. 

There would be two interaction points around the ring, allowing both detectors to operate at the same time, as opposed to the \acrshort{ILC}'s push-pull system. More information on the proposed \acrshort{CEPC} detectors can be found in [reference].

\begin{figure}[h]
	\centering
	\includegraphics[width=1.0\textwidth]{../Pictures/CEPC-Scale.png}
	\caption{Diagram showing the tunnels of the accelerator complex for the proposed \acrlong{CEPC}, with the multiple booster rings and linacs.}
	\label{figure:colliders/CEPC/main}
\end{figure}

\begin{table}[b]
\centering
	\begin{tabular}{ l r l l l l }
	\hline \hline
	\textbf{Parameter} & & \textbf{ILC} & \textbf{CLIC} & \textbf{FCC-ee} & \textbf{CEPC} \\ \hline
	Maximum centre of mass energy & (GeV) & 1400 & 3000 & 400 & 240 \\
	Luminosity & (10\textsuperscript{34}cm\textsuperscript{-2}s\textsuperscript{-1}) & 3.6 & 5.9 & 1.8 & 2 \\
	Power consumption & (MW) & 300 & 589 & TBD & TBD \\
	Bunches per train & & 2450 & 312 & 98 & 50 \\
	Bunch separation & (ns) & 366 & 0.5 & 25 & 25 \\
	Bunch length & (\textmu m) & 250 & 44 & 1160 & 2700 \\
	Bunch population & (10\textsuperscript{10}) & 1.74 & 0.37 & 14 & 37.9 \\

	Accelerating gradient & (MV/m) & 31.5 & 100 & N/A & N/A \\
	Bending radius & (km) & N/A & N/A & 11 & 6.1 \\
	Energy loss per turn & (GeV) & N/A & N/A & 7.55 & 3.11  \\ \hline \hline

%	property & () & x & x & x & x \\ \hline
%	Electron polarisation & (\%) & 80 & 80 & x & TBD \\
%	Positron polarisation & (\%) & 20 & -- & x & TBD \\

	\end{tabular}
	\caption{Summary and comparison of future lepton collider concepts}
	\label{table:colliders/parameters}
\end{table}
