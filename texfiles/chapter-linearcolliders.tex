\chapter{Future Linear Colliders}

\epigraph{Progress is not a straight line.}{An Wang}

In the post-LHC era, the major unanswered questions in particle physics centre around the Higgs boson and its properties, the identification of additional sources of CP-violation that can account for the universe's abundance of matter and paucity of antimatter, and the discovery of physics beyond the Standard Model. There are many investigations into each of these fields that utilise the Large Hadron Collider, or will leverage the upgrades for the high-luminosity LHC (HL-LHC). However, now that the Higgs boson has been identified successfully, one of the most [x] avenues for further research is the construction and operation of a lepton collider with sufficient centre-of-mass energy to produce Higgs bosons. [...]

These are the primary motivations for the construction of future colliders, especially lepton colliders, to succeed the Large Hadron Collider. The two main candidates are the International Linear Collider (ILC) and  the Compact Linear Collider (CLIC). Since both are linear eletron-positron colliders, they share many features, design considerations, and challenges, 

\section{Introduction}
[...]

\section{The physics case for a lepton collider}
[...]

\section{The International Linear Collider}
[...]

The proposed site for the ILC is the Kitakami Highlands region of Iwate prefecture, Japan. [...]

As of writing, the Science Council of Japan (an representative organisation of the Japanese science community) has produced a report expressing that they have not reached a consensus as to whether to support hosting the ILC in Japan. Some  of the reasons cited were concerns over international cost-sharing in the long-term, as well as concerns as to whether the expected scientific outcomes would justify the unprecedented human resource requirements and infrastructure necessary to make the ILC a reality \cite{linearcolliders-scj-report}

The final decision to host the ILC will be made by the Japanese government's Ministry of Education, Culture, Sports, Science and Technology (MEXT), [...]

\subsection{The ILD and SiD detectors}
[...]

One of the unique features of the ILC is the push-pull detector system. This is a moving platform in the chamber housing the interaction point, upon which two detectors can be mounted. The platform can be moved to change which detector is in the beamline, allowing a linear collider to function with multiple detectors. Switching detectors is expected to take [some] hours. This allows the two detectors to specialise for different physics studies and goals, much like the experiments at the Large Hadron Collider at CERN, which is normally not possible with linear colliders. [?] [...]

\subsubsection{The International Large Detector (ILD)}
[...]

\begin{figure}[h]
	\centering
	\includegraphics[width=0.75\textwidth]{../Pictures/SimulatedEvent1.png}
	\caption{Visualisation of a simulated tth event in the ILD. Charged particles can be easily identified by their curved, coiled or spiral paths, and the jets are clearly visible as the light pink and purple areas near the beampipes on either side.}
	\label{figure:colliders/ILD/tth-simulation}
\end{figure}

\subsubsection{The Silicon Detector (SiD)}
[...]

\section{The Compact Linear Collider}
[...]

[...] CLIC would be built beneath the existing LHC ring at CERN, stretching across the French-Swiss border and running parallel to the feet of the Jura mountain range. [...]

As of writing, the CLIC project has been submitted as input for the European Particle Physics Strategy Update, which will decide which projects the CERN collaboration chooses to pursue from 2020  onwards. [...]

CLIC's initial centre-of-mass energy will be 380 GeV, with successive upgrades increasing it to 1.5 TeV and 3 TeV. 