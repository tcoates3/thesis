\chapter{Discussion}
\label{chapter:discussion}

\epigraph{What we know is really very, very little compared to what we still have to know.}{Fabiola Gianotti}

Now that the \acrlong{LHC} has confirmed the existence of a \acrlong{SM}-like Higgs boson, particle physics is at a crossroads. All indications, including the outcome of the recent European Strategy Meeting in Granada, Spain on the 13\textsuperscript{th} May this year \cite{european-strategy}, point to the fact that the construction and operation of a lepton collider at the energy frontier is the clearest path ahead.

This thesis has documented a small portion of the global effort to make a future lepton collider a reality, and to pave the way for the future of the study of high energy particle physics.

\subsubsection*{DQM4hep}
The \acrfull{DQM4hep} tool was introduced and presented in detail, and its role as a generic online monitor for particle physics experiments, especially within the \acrshort{AIDA}-2020 collaboration, was discussed. Development of the important aspect of quality tests was presented, which will allow faster and richer examination of testbeam data with the tool in the future.

An important factor for the uptake of new software like \acrshort{DQM4hep} is accessibility. A large contribution that has been made in aid of this is in the form of full, in-depth documentation. While there was existing developer documentation, and other team members wrote user-facing guides about the operation of the framework and it's structures, it was also necessary to have a much more focused literature aimed at potential users and their immediate experimental needs.

The user guide explaining the structure of analysis modules, and in-depth examples of how to create new ones, will be a critical resource for any new teams attempting to integrate \acrshort{DQM4hep} into their experiment workflows, as well as providing vital documentation for new members of teams who already use the framework. By lowering the barrier to entry, it continues the mission of the \acrshort{DQM4hep} to let the physicists focus on the physics, rather than engineering software solutions to solve the same problems over and over again.

Likewise, the quality test user guide provides an in-depth guide to a relatively new feature, but one that is especially powerful. The quality testing feature is foreseen to be most useful within more mature testbeam environments, where the structure and characteristics of the data has been set, and there are precedents for how data should look. This foreknowledge allows the quality tests to be best leveraged, where they are focused on acting as automated checks on how the testbeam is proceeding.

With these combined with the upgrades to the user interface, the framework will be even easier to use and manipulate, which can only benefit its uptake in the community.

\subsubsection*{\acrshort{AIDA}-2020 and IDEA testbeams}
In the \acrshort{AIDA}-2020 testbeams, understanding and experience of the framework was developing alongside its implementation in experiments. This process was the first time the framework had been applied outside of the environment it was developed in, and the experience acquired was invaluable to understanding how the framework needed to evolve and advance. It also highlighted the important role of documentation and user experience.

Following the experience and expertise acquired in deploying \acrshort{DQM4hep} on the \acrshort{CALICE}-\acrshort{AHCAL} testbeams, this experience was then put to the test using the different and multi-faceted environment of the \acrshort{IDEA} combined testbeam. This formed a `stress-test' of the framework, using multiple different types of detector on the same testbeam, with each having their own datatypes, structures, and idiosyncrasies. \acrshort{DQM4hep} was able to not only monitor all of these detector types, but also proved that it can even be used for online analysis of these detectors.

These implementations have shown that \acrshort{DQM4hep} provides everything that is needed from a generic online monitor and data quality monitor. It can adapt to detectors of many types, timing structures, and triggers. There is even ongoing discussion on the possible use of \acrshort{DQM4hep} in the \acrshort{DAMIC} direct dark matter detection experiment. 

\subsubsection*{Physics studies at the \acrlong{CLIC}}
The performance of the \acrshort{SiD} at the \acrlong{CLIC} was also examined, using simulated physics and detector interactions of the fully hadronic decay channel of the $e^+ e^- \rightarrow t \overline{t} H$ process. 

Combined with a similar study of the semi-leptonic channel, it was found that \acrshort{CLIC} operating at $\sqrt{s}$ = 1.4 TeV using the \acrshort{SiD} detector concept would be able to determine the cross-section of the $e^+ e^- \rightarrow t \overline{t} H$ process with an uncertainty of 7.3\%. The measurement of this cross-section can then be used to directly calculate the top-Higgs Yukawa coupling $y_t$, meaning that the results of this study can also determine the uncertainty of the measurement of this Yukawa coupling directly.

The Yukawa coupling is an important quantity for examining the dynamics of the Higgs mechanism and the associated Higgs sector of the \acrlong{SM}, forming an extremely valuable test of the limits of the \acrshort{SM} and a probe for the presence of elusive new physics. The Yukawa coupling of the top quark represents the most accessible of these couplings, and has implications to both the \acrshort{SM} and many \acrshort{BSM} theories. The result of this study was that the top-Higgs Yukawa coupling could be determined with an uncertainty of 3.8\%, offering an increase over previous studies of this process and a significant increase over similar studies that can be performed at the existing experiments at the \acrlong{LHC}.

The ability of the \acrlong{CLIC} to perform this measurement with an uncertainty at the percent level, using only one-tenth the centre of mass energy and an integrated luminosity seven orders of magnitude lower than the \acrshort{LHC} is a perfect demonstration of the physics case for a lepton collider at the energy frontier.

This analysis is emblematic of the advantages of the significantly cleaner final state of a lepton collider, and the advantages of the particle flow approach in both detector design and analysis. These advantages are shared by all of the proposed lepton colliders.

\subsubsection*{Conclusion}
It is often said that particle physics is something like trying to discern the exact mechanisms, inner workings, and design of a finely-crafted pocketwatch by firing it from a cannon at a wall, and sifting through the resulting debris.

Many of the results of these experiments are pulverised brick, providing little to no insight. The other glimmering remnants are what we're interested in -- the scraps of metal, the fragments of gems, the twisted springs, and the shards of glass. It's these that we use to try to piece together the whole.

This may seem like madness. But we find ourselves in a universe full of an astonishing variety of fascinating pocketwatches, yet utterly devoid of the tools we might use to cautiously open them up and examine their mechanisms in a more calm manner. So it seems the cannon is all we have. \\

In which case, we better get started building a bigger one...

%		%		%		%		%		%
%Particle physics is something like trying to understand the shape, design, age and providence of fine porcelain vases by smashing them into pieces, then employing archaeologists to examine the potsherds. 

%This puts us as physicists in the awkward position of being both the pot-smashers \textit{and} the archaeologists.
%
%The current state of particle physics is something like a child with an old, run-down but functional bike. Our parents keep insisting that there's nothing wrong with the one we have, and that they'll only buy a new one when the old one breaks. So we try to ride the bike to exhaustion -- we cycle it in mud, on broken paths and rocks, do ramps and jumps on it, crash it into cars; all in the hope that it'll finally break so that we can have a shiny new bike with all the bells and whistles.

%But despite our efforts, the old bike remains unphased by this treatment. Sure a few scratches here and there, a burst tyre perhaps, but nothing disasterous. And so we must soldier on with the one we have.