\chapter{Discussion and Conclusions}
\label{chapter:discussion}

\epigraph{What we know is really very, very little compared to what we still have to know.}{Fabiola Gianotti}

%		%		%
%It is often said that particle physics is something like trying to discern the exact mechanisms, inner workings, and design of a fine pocketwatch by firing it from a cannon at a wall, and sifting through the resulting debris. 

%Many of the results are pulverised brick, providing no insight into the watch itself. But the minority that we are interested in are the glimmers of metal, the fragments of gems, the twisted springs, and the shards of glass. It's these that we use to try to piece together the whole.
%		OR		%
%Particle physics is something like trying to understand the shape, design, age and providence of fine porcelain vases by smashing them into pieces, then employing archaeologists to examine the potsherds. 

%This then means that we as physicists are both the pot-smashers \textit{and} the archaeologists.
%		%		%

[...]

The ability of the \acrlong{CLIC} to perform this measurement using only a tenth of the centre of mass energy and an integrated luminosity seven orders of magnitude lower while providing a significantly lower uncertainty is a perfect demonstration of the physics case for a lepton collider at the energy frontier.

This analysis is emblematic of the advantages of the significantly cleaner final state of a lepton collider, and the advantages of the particle flow approach in both detector design and analysis.