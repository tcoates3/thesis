% A database of all acronyms used in the thesis. Format for declaring new acronyms is:
%     \newacronym{code}{acronym}{explanation}

%	%	%

\newacronym{ahcal}{AHCAL}{Analogue Hadron Calorimeter}

\newacronym{daq}{DAQ}{Data Acquisition}

\newacronym{dqm}{DQM}{Data Quality Monitoring}

\newacronym{desy}{DESY}{\textit{Deutches Elektronen-Synchrotron} (EN: German Electron Synchrotron)}

\newacronym{sm}{SM}{The Standard Model}

\newacronym{bsm}{BSM}{Beyond the Standard Model}

\newacronym{cp}{CP}{Charge Parity}

\newacronym{lhc}{LHC}{Large Hadron Collider}

\newacronym{hl-lhc}{HL-LHC}{High Luminosity Large Hadron Collider}

\newacronym{ilc}{ILC}{International Linear Collider}

\newacronym{clic}{CLIC}{Compact Linear Collider}

\newacronym{jinr}{JINR}{Joint Institute for Nuclear Research}

\newacronym{ild}{ILD}{International Linear Detector}

\newacronym{sid}{SiD}{Silicon Collider}

\newacronym{aida}{AIDA}{Advanced Infrastructures for Detectors at Accelerators}

\newacronym{gui}{GUI}{Graphical User Interface}

\newacronym{xml}{XML}{Extensible Markup Language}

\newacronym{siwecal}{SiWECAL}{Silicon Tungsten Electronic Calorimeter}

\newacronym{calice}{CALICE}{Calorimeter for Linear Collider Experiment}

\newacronym{flc}{FLC}{\textit{Forschung mit Lepton Collidern} (EN: Research with Lepton Colliders)}

\newacronym{eudaq}{EUDAQ}{EU [?] Data Acquisition}

\newacronym{bif}{BIF}{Beam Interface}

\newacronym{sps}{SPS}{Super Proton Synchrotron}

\newacronym{adc}{ADC}{Analogue to Digital Conversion}

\newacronym{tdc}{TDC}{Time to Digital Conversion}

\newacronym{lcio}{LCIO}{Linear Collider Input/Output}

\newacronym{mip}{MIP}{Minimum Ionising Particle}

\newacronym{idea}{IDEA}{???}

\newacronym{dwc}{DWC}{Delayed Wire Chamber [?]}

\newacronym{gem}{GEM}{[???]}

\newacronym{sipm}{SiPM}{Silicon Photomultiplier}

\newacronym{csv}{CSV}{Comma-Separated Values}

\newacronym{tmva}{TMVA}{Toolkit for Multi-Variate Analysis}

\newacronym{bdt}{BDT}{Boosted Decision Trees}

\newacronym{qcd}{QCD}{Quantum Chromodynamics}

