\chapter{AIDA-2020 testbeams}

\epigraph{I love fools' experiments. \\I am always making them.}{Charles Darwin}

One of the most important aspects of testing and developing DQM4hep was to ensure that it was as generic as it was intended to be, and this meant deploying and using the framework on physics testbeams.

% Sections at the beginning detailing my contributions in the chapter.

[...]

\section{Introduction}

Regular testbeams were held at the DESY II synchrotron at DESY in Hamburg and at the Super Proton Synchrotron (SPS) at CERN. 

[...]

DQM4hep was largely tested and developed during testbeams of the SiWECAL [...]

\section{CALICE testbeams}
[...]

[...] DQM4hep was used in testbeams of the AHCAL over the course of the years 2016-2018, occurring predominantly at the DESY II facility, with one in May 2017 taking place at the CERN SPS.

\subsection{May 2016 at DESY II}
[...] [This was the three-week one, where we got some real shit done.]

Before and during testbeam, the majority of development for AHCAL-specific analysis modules was undertaken. Prior to this, DQM4hep had only been used on SiWECAL beams, and was untested for other detectors. However, file reader and streamer plugins for the SLCIO data format were already present, which meant that the only part needing work was the analysis modules themselves.

To begin with, two analysis modules were developed: 

\begin{itemize}
	\item \texttt{AHCALChannelSpectra} created a histogram for each channel present in the detector, and filled it with the ADC value for each readout cycle in the whole run, producing a per-channel spectrum of ADCs.
	\item \texttt{AHCALHitmap} created a two dimensional histogram, with each bin representing a channel on the $x$ and $y$ axes, and filled each channel with the ADCs of that channel for the whole event, producing a hitmap. 
\end{itemize}

[...]

Creating the hitmap was nontrivial, as the information coming from the data acquisition and stored in SLCIO format did not have geometric information for each channel, instead only specifying the ``electronics number'' -- a combination of the board number and channel number. To determine the location of any given channel, a mapping was needed. This mapping was not simple -- each board contained sixteen channels, and each layer was formed of four boards tiled together. Further, the numbers of the boards and which order the layers were stacked could also change.

For every testbeam, there was a mapping file that described the position of each board and channel in $(i, j, k)$ co-ordinates. To implement the hitmap analysis module, the module used DQM4hep's libraries for XML parsing to build two functions inside the analysis module class --  \texttt{electronicsToIJK} and \texttt{IJKToElectronics} -- that converted either from electronics number to geometric co-ordinates, or vice versa.

Using this method ensured that, given the geometry file was provided, whatever changes in geometry or set-up of the experiment were always reflected in the hitmaps, and no alteration of hard-coded information or recompilation of the modules was required.

[...]

\subsection{December 2016 at DESY II}
[...]

\subsection{May 2017 at CERN SPS}

During May 2017, testbeam time at CERN's Super Proton Synchrotron (SPS) facility was used for further tests for the AHCAL.

\begin{center}
	[Figure: we have plenty of pictures of the testbeam area and the installation.]
\end{center}

[...]

During this testbeam, development of the analysis modules for DQM4hep had largely been completed and the data format of the detector had been fixed for some time. Because of this, after the initial set-up and verification stages, very little management of the monitoring software was necessary. It was instead used as intended -- a tool for shifters to use to troubleshoot problems with the beam or detectors. It was successfully used to identify dead channels on several of the boards [confirm this].

[...]

\begin{center}
	[Plots from the AHCAL CERN testbeam]
\end{center}
