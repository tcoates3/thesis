\chapter{AIDA-2020 testbeams}

\epigraph{I love fools' experiments. \\I am always making them.}{Charles Darwin}

% Sections at the beginning detailing my contributions in the chapter.

One of the most important aspects of testing and developing DQM4hep was to ensure that it was as generic as it was intended to be, and this meant deploying and using the framework on physics testbeams. DQM4hep was developed during testbeams of the SiWECAL, and its early testing phases were predominantly based on this detector, so it was apparent that it could be used in the originally-intended setting. However, in trying to develop it as a generic monitor, and to satisfy the requirements of a generic data monitoring and quality monitoring tool for AIDA-2020, it was essential that it was tested on other detectors of different types to demonstrate its generic natures. 

[...]

\section{Introduction}

The CALICE testbeams were done with the CALICE collaboration, working within the \textit{Forschung mit Lepton Collidern} (EN: "Research with Lepton Colliders", or FLC) on the Analogue Hadronic Calorimeter (AHCAL) prototype during its development. The AHCAL is a sampling calorimeter formed of steel absorber plates and plastic scintillator tiles, read out by silicon photomultipliers (SiPMs) as active material. One of the important features of the AHCAL is that the prototypes were assembled using techniques suitible for mass production, such as injection-moulding and automated foil-wrapping of the scintillator tiles, as well as the usage of pick-and-place machines for the assembly of layers and their electronics. It also uses power pulsing to reduce power consumption and heat production, rapidly cycling power to be active only when the beam is present, according to the known beam structure. \cite{proceedings-ahcal-prototype} % Citation here isn't working

[...]

Regular testbeams were held at the DESY II synchrotron at DESY in Hamburg, Germany and at the Super Proton Synchrotron (SPS) at CERN in Geneva, Switzerland. [...]

[...] DQM4hep was used as the online monitoring and data quality monitoring tool for the AHCAl beginning in May 2016, and in further testbeams between 2016 and 2018. The majority of these testbeams occurred at the DESY II facility, but two took place at the CERN SPS in May 2017 and June 2018.

\subsection{May 2016 at DESY II}
The testbeam where DQM4hep was deployed for another detector for the first time was during a testbeam of the AHCAL at DESY II during May 2016. The testbeam was to be two weeks in duration, with a one-week setup and preparation period. There were a variety of goals for this long testbeam as a whole, including testing some of the data acquisition hardware and software, developing calibration methods, and to test the deployment of DQM4hep. [...]

% Check that all of these are actually right, then reference the report for it. 

[...] [This was the three-week one, where we got some real shit done.]

Before and during testbeam, the majority of development for AHCAL-specific analysis modules was undertaken. Prior to this, DQM4hep had only been used on SiWECAL beams, and was untested for other detectors. 

However, file reader and streamer plugins for the LCIO data format were already available in the now-deprecated \texttt{dqm4ilc} package, which meant that the framework could open and access the data format already.

\subsubsection{Data format}
The data for the AHCAL was in an LCIO format called LCGenericObject, which is a generic format for use when the existing data formats are not suited. It comprises a header that contains user-defined parameters. These parameters usually include a timestamp, the typename of the object, and a string describing the content of the data and how to parse it. For instance:

% This section commented out because it breaks formatting -- we need a better way to handle it.
%\begin{verbatim}
%--------------- print out of LCGenericObject collection --------------- 
%
%  flag:  0x0
% parameter DAQquality [int]: 1, 
% parameter DataDescription [string]: i:CycleNr,i:BunchXID,i:EvtNr,i:ChipID,i:NChannels,i:TDC14bit[NC],i:ADC14bit[NC], 
% parameter Timestamp [string]: Thu, 25 May 2017 05:38:25 +0200, 
% parameter TypeName [string]: CaliceObject, 
%
% [   id   ] i:Type,i:EventCnt,i:TS_Low,i:TS_High - isFixedSize: false
% --------------------------------------------------------
% [00000852] i:99; i:0; i:0; i:121; i:36; i:13365; i:13383; i:13378; i:13370; i:13336; i:13351; i:13361; i:13365; i:13357; i:13338; i:13345; i:13368; i:13386; i:13380; i:13386; i:13391; i:13382; i:13363; i:13395; i:13342; i:13378; i:13335; i:13327; i:13376; i:13342; i:13373; i:13406; i:13323; i:13361; i:13395; i:13378; i:13365; i:13362; i:13378; i:13384; i:13355; i:12288; i:12288; i:12288; i:12288; i:12288; i:12288; i:12288; i:12288; i:12288; i:12288; i:12288; i:12288; i:12288; i:12288; i:12288; i:12288; i:12288; i:12288; i:12288; i:12288; i:12288; i:12288; i:12288; i:12288; i:12288; i:12288; i:12288; i:12288; i:12288; i:12288; i:12288; i:12288; i:12288; i:12288; i:12288; i:12288;  --------------------------------------------------------
%\end{verbatim}

In this case, the \texttt{TDC14bit[NC]} and \texttt{ADC14bit[NC]} are arrays, each holding a number of elements equal to the \texttt{NChannels} variable, in this case 36. Each element of these arrays corresponds to a single physical scintillator tile within the detector, and identifies which chip it belongs to using \texttt{ChipID}. 

[...]

\subsubsection{Analysis modules}
To begin with, two analysis modules were developed: 

\begin{itemize}
	\item \texttt{AHCALChannelSpectra} created a histogram for each channel present in the detector, and filled it with the ADC value for each readout cycle in the whole run, producing a per-channel spectrum of ADCs.
	\item \texttt{AHCALHitmap} created a two dimensional histogram, with each bin representing a channel on the $x$ and $y$ axes, and filled each channel with the ADCs of that channel for the whole event, producing a hitmap. 
\end{itemize}

[...]

Creating the hitmap was nontrivial, as the information coming from the data acquisition and stored in SLCIO format did not have geometric information for each channel, instead only specifying the ``electronics number'' -- a combination of the board number and channel number. To determine the location of any given channel, a mapping was needed. This mapping was not simple -- each board contained sixteen channels, and each layer was formed of four boards tiled together. Further, the numbers of the boards and which order the layers were stacked could also change.

For every testbeam, there was a mapping file that described the position of each board and channel in $(i, j, k)$ co-ordinates. To implement the hitmap analysis module, the module used DQM4hep's libraries for XML parsing to build two functions inside the analysis module class --  \texttt{electronicsToIJK} and \texttt{IJKToElectronics} -- that converted either from electronics number to geometric co-ordinates, or vice versa.

Using this method ensured that, given the geometry file was provided, whatever changes in geometry or set-up of the experiment were always reflected in the hitmaps, and no alteration of hard-coded information or recompilation of the modules was required.

[...]

\subsection{December 2016 at DESY II}
[...]

\subsection{May 2017 at CERN SPS}

During May 2017, testbeam time at CERN's Super Proton Synchrotron (SPS) facility was used for further tests for the AHCAL.

\begin{center}
	[Figure: we have plenty of pictures of the testbeam area and the installation.]
\end{center}

[...]

During this testbeam, the analysis modules were mature, and represented the majority of the needed online monitoring for the testbeam, especially because the data format of the detector had been fixed for some time. Because of this, after the initial set-up and verification stages, very little management or editing of the monitoring software was necessary. It was instead used as intended -- a tool for shifters to use to troubleshoot problems with the beam or detectors. It was successfully used to identify dead channels on several of the boards [confirm this].

[...]

\begin{center}
	[Plots from the AHCAL CERN testbeam]
\end{center}
