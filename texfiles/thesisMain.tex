\documentclass[a4paper,11pt]{report}

% ???
\usepackage{wrapfig}

% tikz settings
\usepackage{tikz}
\usetikzlibrary{quotes,angles,shapes,snakes,arrows,decorations.markings,decorations.pathmorphing,decorations.markings,decorations.text}
% tikz styles for Feynman diagrams
\tikzset{
    vector/.style={decorate, decoration={snake}, draw=black},
	provector/.style={decorate, decoration={snake,amplitude=2.5pt}, draw},
	antivector/.style={decorate, decoration={snake,amplitude=-2.5pt}, draw},
    fermion/.style={draw=black, postaction={decorate},
        decoration={markings,mark=at position .55 with {\arrow[draw=black]{>}}}},
    fermionbar/.style={draw=black, postaction={decorate},
        decoration={markings,mark=at position .55 with {\arrow[draw=black]{<}}}},
    fermionnoarrow/.style={draw=black},
    gluon/.style={decorate, draw=black,
        decoration={coil,amplitude=4pt, segment length=5pt}},
    scalar/.style={dashed,draw=black, postaction={decorate},
        decoration={markings,mark=at position .55 with {\arrow[draw=black]{>}}}},
    scalarbar/.style={dashed,draw=black, postaction={decorate},
        decoration={markings,mark=at position .55 with {\arrow[draw=black]{<}}}},
    scalarnoarrow/.style={dashed,draw=black},
    electron/.style={draw=black, postaction={decorate},
        decoration={markings,mark=at position .55 with {\arrow[draw=black]{>}}}},
     bigvector/.style={decorate, decoration={snake,amplitude=4pt}, draw},
     line/.style={decorate, draw=black},
}
% tikz styles used for flow diagrams
\tikzstyle{decision} = [diamond, draw, fill=blue!20, 
    text width=4.5em, text badly centered, node distance=3cm, inner sep=0pt]
\tikzstyle{block} = [rectangle, draw, fill=blue!20, 
    text width=8em, text centered, rounded corners, minimum height=4em, minimum width=8em]
\tikzstyle{line} = [draw, -latex']
\tikzstyle{cloud} = [draw, ellipse,fill=red!20, node distance=3cm,
    minimum height=2em]

% epigraph settings, used for quotations
\usepackage{epigraph}
\usepackage{calc}
\newcommand{\mytextformat}{\itshape\epigraphsize}
\newenvironment{mytext}{\mytextformat}{}
\newenvironment{mysource}{\scshape\hfill}{}
\renewcommand{\textflush}{mytext} 
\renewcommand{\sourceflush}{mysource}
\let\originalepigraph\epigraph 
\renewcommand\epigraph[2]%
   {\setlength{\epigraphwidth}{\widthof{\mytextformat#1}}\originalepigraph{#1}{#2}}

% Mathematical symbols, etc.
\usepackage{amsmath}
\usepackage{amsfonts}
\usepackage{amssymb}
\usepackage{adjustbox}
\usepackage{sfmath}

%%%%%%%%%%%%%%%%%%%%%%%%%%%%
% University of Sussex thesis template
%%%%%%%%%%%%%%%%%%%%%%%%%%%%

%%%%%%%%%%%%%%%%%%%%%%%%%%%%
% LINE SPACING
\newcommand{\linespacing}{1.5}
\renewcommand{\baselinestretch}{\linespacing}
%%%%%%%%%%%%%%%%%%%%%%%%%%%%

%%%%%%%%%%%%%%%%%%%%%%%%%%%%
% BIBLIOGRAPHY STYLE
\usepackage{natbib}
% \bibliographystyle{plain} for [1], [2] etc.
\bibliographystyle{apalike}
%%%%%%%%%%%%%%%%%%%%%%%%%%%%

%%%%%%%%%%%%%%%%%%%%%%%%%%%%
% OTHER FORMATTING/LAYOUT DECLARATIONS
% Graphics
\usepackage{graphicx,color}
\usepackage{epstopdf}
\usepackage[british]{babel}
% The left-hand-side should be 40mm.  The top and bottom margins should be
% 25mm deep.  The right hand margin should be 20mm.
% These are the original, intended and asymmetric values:
%		\usepackage[a4paper,top=2.5cm,bottom=2.5cm,left=4cm,right=2cm,headsep=10pt]{geometry}
% Below is the new values, which are preserve the space but make it symmetrical:
\usepackage[a4paper,top=2.5cm,bottom=2.5cm,left=3cm,right=3cm,headsep=10pt]{geometry}
\flushbottom
% Pages should be numbered consecutively thorugh the main text.  Page numbers
% should be located centrally at the top of the page.
\usepackage{fancyhdr}
\fancypagestyle{plain}{
	\fancyhf{}
	% Add "DRAFT: <today's date>" to header (comment out the following to remove)
	\lhead{\textit{DRAFT: \today}}
	%
	\chead{\thepage}
	\renewcommand{\headrulewidth}{0pt}
}
\pagestyle{plain}
%%%%%%%%%%%%%%%%%%%%%%%%%%%%

%%%%%%%%%%%%%%%%%%%%%%%%%%%%
% ANY OTHER DECLARATIONS HERE:

%%%%%%%%%%%%%%%%%%%%%%%%%%%%

%%%%%%%%%%%%%%%%%%%%%%%%%%%%
% HYPERREF
\usepackage[colorlinks,pagebackref,pdfusetitle,urlcolor=blue,citecolor=blue,linkcolor=blue,bookmarksnumbered,plainpages=false]{hyperref}
% For print version, use this instead:
%\usepackage[pdfusetitle,bookmarksnumbered,plainpages=false]{hyperref}
%\usepackage{backref}
%\renewcommand{\backrefpagesname}{Cited on}
%%%%%%%%%%%%%%%%%%%%%%%%%%%%

%%%%%%%%%%%%%%%%%%%%%%%%%%%%
% BEGIN DOCUMENT
\begin{document}
%%%%%%%%%%%%%%%%%%%%%%%%%%%%

%%%%%%%%%%%%%%%%%%%%%%%%%%%%
% PREAMBLE: roman page numbering i, ii, iii, ...
\pagenumbering{roman}
%%%%%%%%%%%%%%%%%%%%%%%%%%%%

%%%%%%%%%%%%%%%%%%%%%%%%%%%%
%% TITLE PAGE: The title page should give the following information:
%%	(i) the full title of the thesis and the sub-title if any;
%%	(ii) the full name of the author;
%%	(iii) the qualification aimed for;
%%	(iv) the name of the University of Sussex;
%%	(v) the month and year of submission.
\thispagestyle{empty}
\begin{center}
\includegraphics[width=6cm]{uslogonew.eps}
\vskip40mm
% TITLE
\huge\textbf{Data acquisition software development and physics studies for a future $e^+e^-$ linear collider}
\vskip5mm
% AUTHOR
\Large\textbf{Tom Coates}
\normalsize
\end{center}
\vfill
\begin{flushleft}
\large
% QUALIFICATION
Submitted for the degree of Doctor of Philosophy \\
University of Sussex	\\
% DATE OF SUBMISSION
March 2019
\end{flushleft}		
%%%%%%%%%%%%%%%%%%%%%%%%%%%%

%%%%%%%%%%%%%%%%%%%%%%%%%%%%
% DECLARATIONS
\chapter*{Declaration}
I hereby declare that this thesis has not been and will not be submitted in whole or in part to another University for the award of any other degree.

% ADDITIONAL DECLARATIONS HERE (IF ANY)

\vskip5mm
Signature:
\vskip20mm
% AUTHOR
Tom Coates
%%%%%%%%%%%%%%%%%%%%%%%%%%%%

%%%%%%%%%%%%%%%%%%%%%%%%%%%%
% SUMMARY PAGE
\thispagestyle{empty}
\newpage
\null\vskip10mm
\begin{center}
\large
\underline{UNIVERSITY OF SUSSEX}
\vskip20mm
% AUTHOR, QUALIFICATION
\textsc{Tom Coates, Doctor of Philosophy}
\vskip20mm
% TITLE
\begin{center}
	\scshape
	\underline{Data acquisition software development and physics studies for}
	\underline{a future $e^+e^-$ linear collider}
\end{center}
\vskip0mm
\vskip20mm
\underline{\textsc{Summary}}
\vskip2mm
\end{center}
% Change line spacing
\renewcommand{\baselinestretch}{1.0}
\small\normalsize
% SUMMARY HERE (300 word limit for most subjects):
[Summary text] [Max. 300 words for most subjects]
%%%%%%%%%%%%%%%%%%%%%%%%%%%%

%%%%%%%%%%%%%%%%%%%%%%%%%%%%
% ACKNOWLEDGEMENTS
\chapter*{Acknowledgements}
\renewcommand{\baselinestretch}{\linespacing}
\small\normalsize
% ACKNOWLEDGEMENTS HERE:

% Fab, Yixuan, Victoria, Remi, Adrian, Katja, David, Matthew, Jenny, ...?

% Academic
I'd first like to thank my supervisor Fabrizio Salvatore, for being exactly as hands-on and hands-off as I needed, when I needed it, and without whose help I wouldn't have been able to do this. Thanks to all of the \emph{Forschung mit Lepton Collidern} group at DESY in Hamburg for [...] but especially to Katja Kr\"{u}ger, Adrian Irles-Quiles (now at LAL Orsay), and R\'{e}mi Et\'{e}.

% Personal
I'd like to thank my highschool physics teachers Mr Coombes and Mr Thomas, who both nurtured my early love for physics (and I'll also take this opportunity to apologise for never doing my homework). [...] 

% Funding
I would like to thank the UK's Science and Technology Funding Council for gratiously providing enough money to prevent me from either starving or freezing to death while performing this research. Thanks also to Linear Collider UK for providing vital funding for travel and a community for linear collider researchers during the dark times, when the UK science bodies were not fully committed to funding linear collider research and when no-one was sure whether either the ILC or CLIC would ever be built; and to the AIDA-2020 collaboration for [...].

%%%%%%%%%%%%%%%%%%%%%%%%%%%%

%%%%%%%%%%%%%%%%%%%%%%%%%%%%
% TABLE OF CONTENTS, LISTS OF TABLES & FIGURES
\newpage
\pdfbookmark[0]{Contents}{contents_bookmark}
\tableofcontents
%\listoftables
%\phantomsection
%\addcontentsline{toc}{chapter}{List of Tables}
%\listoffigures
%\phantomsection
%\addcontentsline{toc}{chapter}{List of Figures}
%%%%%%%%%%%%%%%%%%%%%%%%%%%%

%%%%%%%%%%%%%%%%%%%%%%%%%%%%
% MAIN THESIS TEXT: arabic page numbering 1, 2, 3, ...
\newpage
\pagenumbering{arabic}
%%%%%%%%%%%%%%%%%%%%%%%%%%%%

\chapter{Introduction}


% This kind of breaks the format, because the quote is too long. I want to keep this quote, so we need to look up how to make the epigraph package do a nicer linebreak.
\epigraph{Teaching man his relatively small sphere in creation, it also encourages him by its lessons of the unity of Nature.}{Annie Jump Cannon}{\setlength{\epigraphwidth}{.25\textwidth}}

[...]

\section{The Standard Model}
[...]

\section{The Higgs Boson}
[...]

\section{CP violation in the Higgs sector}
[...]

\section{Data acquisition software and testbeams}
[...] \label{intro}

\chapter{Future Linear Colliders}
\label{chapter:colliders}

\epigraph{Progress is not a straight line.}{An Wang}

In the post-LHC era, the major unanswered questions in particle physics centre around the Higgs boson and its properties, the identification of additional sources of CP-violation that can account for the universe's abundance of matter and paucity of antimatter, and the discovery of physics beyond the Standard Model. There are many investigations into each of these fields that utilise the Large Hadron Collider, or will leverage the upgrades for the high-luminosity LHC (HL-LHC). However, now that the Higgs boson has been identified successfully, one of the most [x] avenues for further research is the construction and operation of a lepton collider with sufficient centre-of-mass energy to produce Higgs bosons. [...]

These are the primary motivations for the construction of future colliders, especially lepton colliders, to succeed the Large Hadron Collider. The two main candidates are the International Linear Collider (ILC) and  the Compact Linear Collider (CLIC). Since both are linear eletron-positron colliders, they share many features, design considerations, and challenges, 

\section{Introduction}
[...]

\section{The physics case for a lepton collider}
[...]

Measurements of particles and their properties will be made significantly easier and thus more precise in the cleaner final state environment of a lepton collider, allowing for precision measurements of the Standard Model, putting better limits on the existence of new physics. It will also allow high-precision measurements of the properties of the Higgs boson, allowing linear colliders to determine their properties very precisely. Multiple BSM models rely upon properties of the Higgs being different from those predicted by the Standard Model, and 

%Bydd mesuron o ronynnau a'u nodweddion yn haws mawr a felly manolach yn yr ystad gorffenol glanach o wrthdarwr llinellol, gadael mesuruon manoldeb o'r Model Safonol, gwneud terfynau gwell ar gyfer y bodolaeth o ffiseg newydd. Bydd hefyd gadael uchel-manoldeb mesuron o'r nodweddion o'r boson Higgs, gadael gwrthdarwr llinellol i benderfynu eu nodweddion yn manwl iawn. Lluos o fodelau Tros y Model Safonol (TMS) yn sefyll y bod y nodweddion o'r Higgs yn gwahanol na'r rheiny wedi daroganu gan y Model Safonol.

\section{The International Linear Collider}

The International Linear Collider (ILC) is a proposed high-luminosity linear electron-positron collider, designed to have an initial energy of between 200-500 GeV, upgradable to 1 TeV at a later date. The ILC and its detectors are designed with the intention of becoming a "Higgs factory" -- producing large numbers of Higgs bosons to allow more detailed study of these particles.

There were a number of proposed sites for the ILC, including CERN in Geneva, DESY in Hamburg, and JINR near Moscow. Due to the 2008 economic crisis, the United States and United Kingdom severely cut funding for linear collider projects, and this resulted in Japan stepping up to offer to host the collider in the Kitakami Highlands region of the Iwate prefecture. This is partially due to a commitment the Japanese government made in [year] to cover half of the cost of construction, commissioning and operation if it was hosted within Japan.

However as of writing, a report from the Science Council of Japan (a representative organisation of the Japanese science community) released in 2019 expressed that they have not reached a consensus as to whether to support hosting the ILC in Japan. Some  of the reasons cited were concerns over international cost-sharing in the long-term, as well as to whether the expected scientific outcomes would justify the unprecedented human resource requirements and infrastructure necessary to make the ILC a reality \cite{linearcolliders-scj-report}.

The final decision to host the ILC will be made by the Japanese government's Ministry of Education, Culture, Sports, Science and Technology (MEXT), [...]

\subsection{The ILD and SiD detectors}
[...]

One of the unique features of the ILC is the push-pull detector system. This is a moving platform in the chamber housing the interaction point, upon which two detectors can be mounted. The platform can be moved to change which detector is in the beamline, allowing a linear collider to function with multiple detectors. Switching detectors is expected to take [some] hours. This allows the two detectors to specialise for different physics studies and goals, much like the experiments at the Large Hadron Collider at CERN, which is normally not possible with linear colliders. [?] [...]

\subsubsection{The International Large Detector (ILD)}
[...]

\begin{figure}[h]
	\centering
	\includegraphics[width=0.75\textwidth]{../Pictures/SimulatedEvent1.png}
	\caption{Visualisation of a simulated tth event in the ILD. Charged particles can be easily identified by their curved, coiled or spiral paths, and the jets are clearly visible as the light pink and purple areas near the beampipes on either side.}
	\label{figure:colliders/ILD/tth-simulation}
\end{figure}

\subsubsection{The Silicon Detector (SiD)}
[...]

\section{The Compact Linear Collider}
[...]

[...] CLIC would be built beneath the existing LHC ring at CERN, stretching across the French-Swiss border and running parallel to the feet of the Jura mountain range. [...]

As of writing, the CLIC project has been submitted as input for the European Particle Physics Strategy Update, which will decide which projects the CERN collaboration chooses to pursue from 2020  onwards. [...]

CLIC's initial centre-of-mass energy will be 380 GeV, with successive upgrades increasing it to 1.5 TeV and 3 TeV.  \label{colliders}

\chapter{Data acquisition software}

\epigraph{Before software can be reusable \\it first has to be usable.}{Ralph Johnson}

Data acquisition is a critical component of all particle physics experiments across all stages of technological readiness, from the very beginning of hardware testing in tabletop experiments to full-scale international experiments like the Large Hadron Collider. 

In the modern era of particle physics, the interplay of hardware and software at minscule timescales drives everything, and almost all results are highly dependent upon the speed and efficiency of the electronics and computer systems that extract data from the detectors. A massive quantity of work goes into creating, testing and optimising the systems that will acquire, process, sort and transport data before it is ever seen by the physicist operating the experiment.

Of particular interest in this thesis is the data acquisition software during the development phase, where individual detector subcomponents are undergoing prototyping and testing. These development and iteration cycles are tied closely to testbeam facillities such as the Super Proton Synchrotron (SPS) at CERN and the DESY II synchrotron at DESY. At this point in the development cycle, the detectors are beginning to take shape and this is where data acquisition (or DAQ) becomes an important consideration. 

In addition to this, the data acquisition solutions used during the testbeam phase of detector development is likely to inform the final data acquisition solution, either directly by evolving into the final software, or indirectly by identifying and evaluating the particular features or challenges of the subdetector components that the software must take into account or accommodate.

During this stage, each individual detector component -- such as a vertex tracker or hadronic calorimeter -- will be developed by small teams, and the natural tendency is for each of these groups to set their own standards and develop their own tools, prioritising the features that are important to their specific case. However, in the past this approach has generated a variety of \textit{ad hoc} solutions for testbeam software, many of which cannot be applied outside of their original scope. The also results in wasted effort and time, as different teams implement the same solutions anew for each subdetector.

One of the aims of the AIDA-2020 project is to improve this situation by developing generic and reusable software tools for testbeams and particle physics experiments.

\subsubsection{The AIDA-2020 project}
The AIDA-2020 project is an EU-funded research programme for developing infrastructure and technologies for particle physics detector development and testing, comprising 24 member countries and lead by CERN.

The overarching goal of the project is to develop common infrastructures and tools for physics testbeams, and software is one such important tool. By creating a suite of tools that are designed with a variety of uses in mind, the amount of effort and development time necessary to plan and implement data acquisition and monitoring setups can be significantly reduced or eliminated, speeding up the planning and deployment of physics testbeams. This allows more science to be done faster. The two tools within AIDA-2020 that facillitate this are EUDAQ and DQM4hep, discussed in more detail below.

\section{EUDAQ}
[...]

\section{DQM4hep}
Data Quality Monitoring for High-Energy Physics (abbreviated DQM4hep) is an online monitoring and data quality monitoring framework developed for physics testbeams for high-energy and particle physics. It is designed to be able to fulfil the requirements of monitoring for physics testbeams in a generic way. The structure of the program allows for independent components of the framework to be used, not used, or exchanged, by isolating each function of the program into specific and independent processes. The components that are specific to particular users -- the file readers, event streamers, and analysis and standalone modules -- are written in standard C++ code, meaning they are capable of performing any data unpacking, processing or analysis that is necessary. The framework then handles packaging this information in a useful way and networking to transmit it to where it is needed, meaning that the user does not have to worry about the mechanics of data storage, serialisation or transmission. It also means that the framework does not need special rules for handling particular datatypes, allowing it to handle \emph{anything} that can be packed into, decoded from, and accessed by normal C++ methods. This results in a framework that is able to deal with any kind of data, including user-defined data types, making it more flexible, portable and easily reusable.

\subsection{Prerequisites and dependencies}

DQM4hep is a C++ application, written in the C++11 standard, that can run on any Linux distribution. The only requirements for installation are a compiler compliant with the C++11 standard, cmake 3.4 or higher, and ROOT 6. All other prerequisites or dependencies are downloaded and compiled by the framework's installer. 

[...]

\subsection{Programming paradigms and structure}

The two core principles for DQM4hep are genericity and modularity. The core of the system is based on a plugin system to allow shared libraries to be loaded and hook classes for further use. \cite{aida2020-milestone-dqm4hep}

[...]

\begin{figure}[h]
	\centering
	\includegraphics[width=0.95\textwidth]{../Pictures/GlobalArchitectureDiagram.pdf}
	\caption{The global online architecture of DQM4hep.}
	\label{figure:daq/dqm4hep/architecture}
\end{figure}

\subsection{Visualisation and graphical user interface}
As of writing, the graphic user interfaceand visualisation suite is still under active development.  Previous versions of DQM4hep have utilised the Qt framework for GUI and visualisation.

The decision to remove the Qt-based GUI from the framework was because of integration with ROOT -- running DQM4hep previously required an installation of ROOT that had the \texttt{--enable-Qt} flag, and the majority of ROOT installations in remotely-accsible file systems based at CERN and DESY (which are heavily used for analysis and testbeams) were not compiled with this flag. 

The removal of the Qt-based UI however allows for greater freedom with UI development. The intended goal is to have a browser-based UI, removing dependency on specific software frameworks, allowing it to function on any device. [...]

[...]

\subsection{Data quality testing}
One of the important areas of DQM4hep that was not yet completed was data quality monitoring, which is an array of tests or programs that assess the data being taken in real time to allow testbeam operators and shifters without detailed knowledge of the hardware, software, or physics to determine whether the device under test is performing as intended, and to quickly identify and address any errors or inconsistencies. Data quality monitoring (DQM) uses a variety of methods for measuring the `quality' or `goodness' of data, mainly relying upon statistical or comparative methods.

DQM4hep did not have any infrastructure to support data quality tests, but this was added during the core refactoring for the release version [?]. Once this was in place, a variety of data quality tests were developed and implemented, ranging from basic tests, such as comparing the mean of a data sample against predefined values of mean and standard deviation, to more complex tests, such as the Kolmogorov-Smirnov test, which is a comparison between a sample and a reference histogram.

% Here we should have figures demonstrating the qtests. Maybe one mean/stddev (if we can find visual information for this), and one comparison to reference, e.g. Kolmogorov.

[...]

\section{Adaptation to other detectors}

[...]

To utilise DQM4hep with any new detector, either two or three new plugins must be created depending on whether the detector is to be monitored online, offline, or both.

If the data is to be monitored offline, then a file reader plugin must be written. If the data is to be monitored online, then a streamer plugin must be written. Both of these plugins are similar in structure and differ only on where they get the data from -- a file reader loads a file from disk, whereas a streamer loads it from the data acquisition system. Once the information is accessible from the reader plugin, it packages the data into events, and emits them to the framework's network handling to be received by any other plugins that are listening for them. The incoming data can be of any type, since the methods for reading it are provided in the reader plugin itself. Previously, reader plugins have been created to read binary files, raw text files, SLCIO data files, and ROOT trees.

An analysis module is a type of plugin which takes data that is packaged into events by a reader plugin and performs some analysis on it. Each analysis module can only read events from one reader, but [...] 

Writing new readers and analysis modules is relatively simple, especially if the data has already been packaged into well-structured formats such as ROOT or SLCIO. [...]

% Place an example file reader here, well-commented?

\section{Integration with EUDAQ}
[...]

\section{Documentation and user guide}

One of the biggest hurdles to promoting a new framework is the lack of understanding of it's installation, deployment, and usage. Many research teams will continue to use their existing software solutions, which may be suboptimal or difficult to use, according to the principle of `better the devil you know than the devil you don't`. The first step to overcoming this is to produce clear, readable and complete documentation across the entire range of features the framework has.

[...]

Many particle physics software frameworks provide documentation in the form of Doxygen, a documentation generator tool. Doxygen is extremely useful, as the documentation for functions and objects is written within the code itself, ensuring that documentation is written as the code is written. Doxygen also automatically compiles a documentation guide using HTML, automatically listing the relationships between objects, structures, functions, etc. that can be compiled and viewed locally, or hosted on the internet as a reference.

One issue with Doxygen documentation is that is not holistic. Doxygen compiles documentation for individual structures of the program, not of the program as a whole. Doxygen is extremely useful for developers, as well as alread-experienced users. However it doesn't aid new users in learning to use   a piece of software, 

[...]

An important aspect with this holistic documentation was that it was to be very distinct from the developer documents (i.e. Doxygen). These were to be a set of guides and walkthroughs for common procedures, intended for \emph{users} of the framework with little to no interest in the mechanics of it.

[...]

\subsection{File reader plugins}
[...]

\subsection{File streamer plugins}
[...]

\subsection{Analysis and standalone modules}
[...]
 \label{daq}

\chapter{AIDA-2020 testbeams}

\epigraph{I love fools' experiments. \\I am always making them.}{Charles Darwin}

[...]

\section{Introduction}
[...]

\section{The CALICE-AHCAL}
[...]

\section{The CALICE-SiWECAL}
[...]

\section{DREAM combined testbeam}
[...]

The combined testbeam comprised four separate dectors: a calorimeter, a muon detector and preshower, a drift chamber, and a silicon photomultiplier. [...]

Importantly, none of these detectors or the teams responsible for their construction and operation were part of AIDA-2020, which was useful as a testbed for the generic nature of the DQM4hep framework - previous testbeams had only used AIDA-2020 detectors, many of which used filetypes or structures defined within the collaboration. By attempting to use DQM4hep to monitor non-AIDA-2020, it was possible to test that the design of the framework was truly generic and adaptable to any kind of detector with any filetype.

[...] \label{testbeams}

\chapter{Physics studies for the Compact Linear Collider}
[...]

\section{Physics generation and samples}
[...]

\section{Detector models}
[...]

\section{Previous work}
[...]

\section{Sensitivity to cross-sections and Yukawa coupling}

[...] The combined uncertainties, based on both the semi-leptonic and fully hadronic channels, are:

Cross-section: $$\Delta\sigma = 7.30\% $$

Top-Higgs Yukawa coupling: $$ \frac{\Delta g_{tth}}{g_{tth}} = 3.86\% $$

\section{Determination of sensitivity to CP-violation}
One of the primary goals for CLIC is to determine any additional sources of CP-violation within	 the Standard Model, or any sources that may indicate BSM physics. The top-Higgs Yukawa coupling is one of the processes that can investigate this, a Higgs bosons with an odd CP quantum number (or ``CP-odd'' Higgs bosons) may be an indication of BSM physics, and determining the mixing ratio or angle between these Higgs bosons is an important result.

\subsection{Up-down asymmetry}
The up-down asymmetry is a simple observable that has already been identified for investigating CP-violation in the $tth$ process. It is found by defining a plane from the vectors of the incoming electron and produced antitop, then finding the ratio of top quarks that are emitted above and below this plane. If there is no CP-violation in this process, the ratio will be even.

Using the up-down asymmetry as an observable requires that the $W^+$ and $W^-$ can be distinguished from each other, and has thus far only been used for the semi-leptonic decay channel. In the semi-leptonic case, the lepton produced by the decay of one of the W bosons identifies its charge, and thus the charge of the top quark that it has decayed from. While this method was not previously possible in the fully hadronic case, I discuss a method for applying it in Section \ref{physics/CP/jetcharge}, using jet charge determination.

\subsection{Jet charge determination} \label{physics/CP/jetcharge}
[...]

However, the jet charge can be determined by summing the charges of all particles in the jet weighted by their $p_T$, normalised by the $p_T$ of the entire jet:
		
\begin{displaymath}
	\mathcal{Q}_\kappa^i = \frac{1}{(p_T^{jet})^\kappa} \sum_{j \in jet} Q_j (p_T^j)^\kappa
\end{displaymath}

Where $\kappa$ is some parameter between 0 and 1, typically set to 1. The efficiency [?] of this method is strongly dependent upon the efficiency of the jet reconstruction, which means the choice of jet clustering algorithm and its parameters are important. Previous analyses of $tth$ events have used the $k_t$ algorithm, as the relative difference between the jet shapes is more important than their absolute shapes, so other algorithms do not provide any benefits.

[...]

For improvements to the jet clustering algorithm, the Valencia algorithm gives better performance within the cleaner final states of a lepton collider, and many analyses are now transitioning to using the Valencia or Durham [?] algorithms.

\subsection{Results}
[...] \label{physics}

\chapter{Discussion and Conclusions}
\label{chapter:discussion}

\epigraph{What we know is really very, very little compared to what we still have to know.}{Fabiola Gianotti}

[...] \label{discussion}

%%%%%%%%%%%%%%%%%%%%%%%%%%%%
% BIBLIOGRAPHY
\clearpage
\phantomsection
\addcontentsline{toc}{chapter}{Bibliography}
\bibliography{bib}
%%%%%%%%%%%%%%%%%%%%%%%%%%%%

%%%%%%%%%%%%%%%%%%%%%%%%%%%%
% START APPENDICES
\appendix
%%%%%%%%%%%%%%%%%%%%%%%%%%%%

\chapter{Code} \label{appenddix/code}

\begin{verbatim}
10 PRINT "LOOK AROUND YOU"
20 GOTO 10
\end{verbatim}

%%%%%%%%%%%%%%%%%%%%%%%%%%%%
% END DOCUMENT
\end{document}
%%%%%%%%%%%%%%%%%%%%%%%%%%%%